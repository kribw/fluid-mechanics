% Article type supporting font formatting
\documentclass[a4paper,11pt]{extarticle}

% Define .tex file encoding
\usepackage[utf8]{inputenc}

% Norwegian language support
%\usepackage[norsk]{babel}

% Sample text
\usepackage{lipsum}  

% Margin defining package
\usepackage{geometry}
\geometry{a4paper,margin=2cm}

% Better bibliography 
\usepackage[round]{natbib}

% For use of graphics in document
\usepackage{graphicx}

% Allows \begin{figure}[H]
\usepackage{float}

% For use of subfigures in document
\usepackage{subcaption}

% Allows easy writing of algorithms
\usepackage[ruled]{algorithm2e}

% Math related tools and functions
\usepackage{mathtools}

% Allows math in tex file 
\usepackage{amsmath}

% Allows math symbols in tex file
\usepackage{amssymb}

% Allows use of physics shortcut functions
%\usepackage{physics}

% For quotations
\usepackage{csquotes}

% Allows multi-line comments in tex file
\usepackage{verbatim}

% Verbatim env with LaTeX commands
\usepackage{alltt}

% Allows more control over tables
\usepackage{tabulary}

% Ads ul which allows line breaks while underlining text.
\usepackage{soul}

% Adds labeling list to the report
\usepackage{scrextend}
\addtokomafont{labelinglabel}{\ttfamily}

% Allows number referencing the last page, used in footer
\usepackage{lastpage}

% No indentation for the paragraphs, set distance between paragraphs
\usepackage{parskip}[skip=6pt]

% Adds "Appendix" before numbering for sections in appendix, use appendices env
\usepackage[title,toc,titletoc]{appendix}

% Change caption size and style
\usepackage[font=footnotesize,labelsep=period]{caption}

% Makes matrices look square-ish
\renewcommand*{\arraystretch}{1.5}

% Change date style
\usepackage[yyyymmdd]{datetime}
\renewcommand{\dateseparator}{-}

% Necessary for defining colours
\usepackage{xcolor}
\definecolor{linkgreen}{rgb}{0,.5,0}
\definecolor{linkblue}{rgb}{0,0.25,1}
\definecolor{linkred}{rgb}{.5,0,0}
\definecolor{blue}{rgb}{.13,.13,1}
\definecolor{green}{rgb}{0,.5,0}
\definecolor{red}{rgb}{.9,0,0}

% Hyperlinks in document
\usepackage{hyperref}
\hypersetup{
  colorlinks=true,     % True for colored links
  linktoc=all,         % True for table of contents links
  linkcolor=linkblue,  % Colour for links
  urlcolor=linkgreen,  % Colour for URLs
  citecolor=linkred    % Colour for citations
}

% Listing package for code examples
\usepackage{listings}         
\lstset{
  language=C++,                       % Set language to C++
  showspaces=false,                   % Don't show space chars
  showtabs=false,                     % Don't show tab chars
  breaklines=true,                    % Break long lines of code
  showstringspaces=false,             % Don't show spaces in strings
  breakatwhitespace=true,             % Break at white space only
  commentstyle=\color{green},         % Set colour for comments
  keywordstyle=\color{blue},          % Set colours for keywords
  stringstyle=\color{red},            % Set colour for strings
  basicstyle=\footnotesize\ttfamily,  % Set basic style
  tabsize=4                           % Set tabsize
}

% Allows editing of section headers
\usepackage{titlesec}
\titleformat*{\section}{\normalsize\bfseries}
\titleformat*{\subsection}{\normalsize\itshape\bfseries}
\titleformat*{\subsubsection}{\normalsize\bfseries}

% Change look and feel of abstract
\usepackage{abstract}
\setlength{\absleftindent}{0mm}
\setlength{\absrightindent}{0mm}
\renewcommand*\abstractname{\flushleft\normalsize\textbf{Abstract}\hfill}

% Allows editing of header and footer data
\usepackage{fancyhdr}
\fancypagestyle{plain}{
  \fancyhf{}
  \renewcommand{\headrulewidth}{0pt}
  \rfoot[R]{\footnotesize Page \thepage\ of \pageref*{LastPage}}
}
\pagestyle{fancy}
\fancyhf{}
\chead{\footnotesize Kristoffer Berg Wilhelmsen, Project in Fluid Mechanics, UiT Narvik 2022}
\rfoot{\footnotesize Page \thepage\ of \pageref*{LastPage}}
\setlength{\headheight}{20pt}
\setlength{\footskip}{20pt}

% Referencing, last for compatibility reasons
\usepackage[noabbrev,nameinlink]{cleveref}

%%%%%%%%%%%%%%%%%%%%%%%%%%%%%%%%%%%%%%%
%%      Title, Author, and Date      %%
%%%%%%%%%%%%%%%%%%%%%%%%%%%%%%%%%%%%%%%
\title{Fluid Mechanics}
\author{Kristoffer Berg Wilhelmsen}
\date{\parbox{\linewidth}{\centering
    \textit{\small UiT - The Arctic University of Norway, P.O. Box 385, N-8505 Narvik, Norway}\endgraf\bigskip
    \small Submitted \today
}}
\providecommand{\keywords}[1]{\flushleft\textit{\small{Keywords:}} #1}

%%%%%%%%%%%%%%%%%%%%%%%%%%%%%%%%%%%%%%%
%%           Start document          %%
%%%%%%%%%%%%%%%%%%%%%%%%%%%%%%%%%%%%%%%
\begin{document}

%%%%%%%%%%%%%%%%%%%%%%%%%%%%%%%%%%%%%%%
%%   Create the main title section   %%
%%%%%%%%%%%%%%%%%%%%%%%%%%%%%%%%%%%%%%%
\maketitle

%%%%%%%%%%%%%%%%%%%%%%%%%%%%%%%%%%%%%%%
%%      Abstract for the report      %%
%%%%%%%%%%%%%%%%%%%%%%%%%%%%%%%%%%%%%%%
\noindent\rule{\linewidth}{.5pt}
\begin{abstract}
  TL;DR

  \keywords{\small{Fluid Mechanics}; \small{Godot}; \small{Flow-3d}}
\end{abstract}
\rule{\linewidth}{.5pt}

%%%%%%%%%%%%%%%%%%%%%%%%%%%%%%%%%%%%%%
%%  The main content of the report  %%
%%%%%%%%%%%%%%%%%%%%%%%%%%%%%%%%%%%%%%

\section{Introduction}
The AW101, domestically known as the Search \& Rescue (SAR) Queen in Norway, is a helicopter specialized for use in rescue missions \citep{ProjectDescription}. It is a replacement for Norway's previous rescue helicopter, known as Sea King. As the SAR Queen has been built to fly long distances, withstand the arctic climate, strong winds and rough seas, it is essentially "The Terminator" of rescue helicopters.

Despite the SAR Queen being a phenomenal helicopter for its intended area of use, it unfortunately also has its downsides. The downdraft created by the helicopter is much more significant than its predecessor's, and can be a potential danger for humans in proximity.

The objective of the work is to analyze the given flow problem. A pre-defined numerical simulation performed by professor Per-Arne Sundsbø using recorded sensor data from a real landing has been provided, along with the 3D models used in the simulation. By applying modern visualization techniques, the analysis will look to get a better understanding of the flow problem.

%The objectives of the work are to perform post-processing on data obtained from a simulation.

% \lipsum[2-4]

\section{Material \& Methods}
Various pieces of software have been used in order to perform the analysis of the given problem. This section describes what software was used and how.

\subsection{Blender}
Blender is a free and open source software used to create and animate 3D models \citep{blender}. To be able to use the supplied 3D models, their format must be compatible with the visualization we intend on using. In this case we will use Godot to create visualizations, which does not currently support the supplied \textit{stl} format. Godot's preferred format for 3D models is \textit{glTF} (.glb file extension). Converting the models can easily be done by importing them to Blender, and exporting them to the desired format.

\subsection{Python}
To

\subsection{Godot}
Godot is a game engine which as the name suggests, is primarily used for game development. However, we can also use Godot to create animations.

\subsection{OBS Studio}
Deez

\section{Results \& Discussion}
What results did we get!?

\section{Conclusions}
What is the conclusion?

\section{Acknowledgements}
Special thanks to my brother Kenneth Wilhelmsen for feedback, advice and inspiration on possible ways to approach the task. I would also like to thank professor Per-Arne Sundsbø for providing the Flow3D simulation that was used in the analysis.

\section{References}
\begingroup
\def\section*#1{}
\bibliographystyle{apalike}
\bibliography{References}
\endgroup


%%%%%%%%%%%%%%%%%%%%%%%%%%%%%%%%%%%%%%%
%%      Appendix for the report      %%
%%%%%%%%%%%%%%%%%%%%%%%%%%%%%%%%%%%%%%%

% New Page for appendix
\pagebreak
\begin{appendices}

  % Reset counter for figures, equations, and tables.
  \setcounter{figure}{0}
  \setcounter{table}{0}
  \setcounter{equation}{0}

  % Set numbering for figures, equations and tables A.1, A.2, etc.
  \renewcommand\thefigure{\thesection.\arabic{figure}}
  \renewcommand\thetable{\thesection.\arabic{table}}
  \renewcommand\theequation{\thesection.\arabic{equation}}

\end{appendices}

\end{document}